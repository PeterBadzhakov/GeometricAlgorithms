\documentclass{article}
\usepackage[bulgarian]{babel}
\usepackage{amsmath}
\usepackage{amssymb}
\usepackage{listings}
\usepackage{graphicx}
\usepackage{hyperref}

\lstset{basicstyle=\ttfamily,
  showstringspaces=false,
  commentstyle=\color{red},
  keywordstyle=\color{blue}
}

\begin{document}

\title{Квадратични B-сплайнове с контролни точки на де Бор: Подробно математическо обяснение и имплементация на C}
\author{Петър Баджаков, Ф.Н. 4MI3400199}
\date{08.07.2023}
\maketitle

Това е документацията за проекта, качен на \href{https://github.com/PeterBadzhakov/GeometricAlgorithms/}{GitHub}.

\section{Въведение}
Този документ засяга математическите концепции, стоящи зад създаването на квадратични B-сплайни с помощта на контролни точки на де Бор, и съответната имплементация на C с използването на библиотеката Simple DirectMedia Layer (SDL) за графичния интерфейс.

\section{Квадратични B-сплайни}
B-сплайните са полиномиални функции, дефинирани по части, и принадлежат към семейството на кривите на Безие. Степента на полинома определя гладкостта и гъвкавостта на кривата. Квадратичните B-сплайни, които са от степен 2, са параболични секции, свързани по такъв начин, че да се гарантира непрекъснатост и гладкост.

За даден набор от контролни точки, $\mathbf{P}_0, \mathbf{P}_1, ..., \mathbf{P}_{n}$, крива на B-сплайн е дефинирана като сума с тегла от тези контролни точки, където теглата се определят от базисните функции на B-сплайн, означени с $B_{i,k}(t)$:

\begin{equation}
S(t) = \sum_{i=0}^{n} B_{i,k}(t) \mathbf{P}_i
\end{equation}

където $k$ е степента на сплайна.

За квадратичен B-сплайн ($k=2$), тези базисни функции гарантират, че резултатът ще се състои от квадратични функции, дефинирани по части.

\section{Базови функции}
Базисните функции на B-сплайн, $B_{i,k}(t)$, могат да бъдат дефинирани рекурсивно по следния начин:

\begin{equation}
B_{i,1}(t) = 
    \begin{cases}
    1, & \text{ако } t_i \leq t < t_{i+1} \\
    0, & \text{иначе}
    \end{cases}
\end{equation}

\begin{equation}
B_{i,k}(t) = \frac{t - t_i}{t_{i+k-1} - t_i} B_{i,k-1}(t) + \frac{t_{i+k} - t}{t_{i+k} - t_{i+1}} B_{i+1,k-1}(t)
\end{equation}

Обърнете внимание, че тези функции изискват последователност от възлови стойности, $t_0, t_1, ..., t_{n+k}$. В този проект използваме равномерни B-сплайни, където векторът на възелите е последователност от цели числа.

\section{Алгоритъм на де Кастелжо}
Въпреки че алгоритъмът на де Бор често се използва за B-сплайнове, в този проект използваме алгоритъма на де Кастелжо поради неговата простота и по-добра числова стабилност. 

Дадени са контролни точки $\mathbf{P}_0, \mathbf{P}_1, ..., \mathbf{P}_n$, рекурсивната формула на алгоритъма на де Кастелжо е:

\begin{equation}
\mathbf{P}_{i,j}(t) = 
    \begin{cases}
    \mathbf{P}_i, & \text{ако } j = 0 \\
    (1-t)\mathbf{P}_{i,j-1}(t) + t\mathbf{P}_{i+1,j-1}(t), & \text{ако } j > 0
    \end{cases}
\end{equation}

където $\mathbf{P}_{i,j}(t)$ представлява $i^{th}$ контролната точка на $j^{th}$ стадий от рекурсията. 

В този код се имплементира този алгоритъм с двумерен масив.

\section{SDL и имплементация на GUI}
Графичният интерфейс, имплементиран чрез библиотеката SDL, е частта от приложението, осигуряваща потребителския интерфейс за добавяне на контролни точки и визуализацията на резултатния B-сплайн.

Потребителят взаимодейства с GUI чрез кликване на екрана. Всеки клик добавя контролна точка на кликнатото място. Когато броят на контролните точки надвишава степента на сплайна, той се изчислява и показва в реално време. Предоставят се бутони за нулиране на контролните точки и за изход от приложението.

\section{Заключение}
Квадратичните B-сплайни, конструирани с контролни точки на де Бор, имат много приложения в графиките и моделирането, предоставяйки надежден инструмент за генериране на гладки криви от множество контролни точки. Математиката и кодът, предоставени в този документ, предоставят подробно обяснение за стъпките, свързани с генерирането на тези сплайни и визуализирането им чрез GUI.

\end{document}

